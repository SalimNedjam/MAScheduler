\section{Background}

\subsection{User: glibc et mutex}
La glibc est une librairie qui offre un panel de fonctionnalités. 
Celle des verrous (mutex) nous intéresse particulièrement: elle permet 
d'abstraire aux programmeurs l'utilisation complexe des verrous côté kernel.

Il existe plusieurs types de mutex proposés à l'utilisateur, voici une liste non exhaustive:
\begin{enumerate}
	\item TIMED: type par défaut, prend le verrou de manière bloquante.
	\item RECURSIVE: ne tente pas de prendre le verrou si celui-ci est
	déjà détenu par le processus courant.
	\item ADAPTIVE: correspond à la méthode de prise de verrou 'trylock',
	au bout d'un certain nombre d'échec la prise de verrou se fait
	bloquante.
\end{enumerate}
Le choix du type de mutex se fait lors de l'initialisation du verrou.
On peut combiner ces types externes avec des types internes qui permettent de
modifier le comportement du verrou jusqu'au kernel, notamment les types:

\begin{enumerate}
	\item ROBUST: type par défaut, si le propriétaire du verrou meurt sans
	le relâcher la prochaine demande à se verrou sera accepté sans appel
	système.
	\item PI (Priority Inheritance): la priorité du processus détenant le
	verrou est augmenté à la plus grande priorité parmi les processus
	bloqués sur le verrou.
\end{enumerate}
Ces deux types insèrent le \textit{pid} du propriétaire du verrou
directement dans la valeur du verrou. Ainsi, lors des appels systèmes le kernel
peut connaître le \textit{pid} du propriétaire du verrou avec une simple
opération de flag sur la valeur passée.

En tenant compte de ces éléments on peut donc en déduire que la modification 
de la partie glibc n'est pas nécessaire. En effet, le \textit{pid} du 
propriétaire du verrou étant déjà communiqué il nous suffit de le récupérer 
côté kernel. Ainsi, cela permet à notre future solution de ne pas dépendre de 
modification ou de décoration d'une fonctionnalité de la glibc, et d'éviter
des dépendances dur à maintenir entre la glibc et le kernel

\subsection{Kernel: futex}

La glibc peut communiquer avec le kernel à travers des appels systèmes pour effectuer certaines tâches.
Un appel système est coûteux, ce qui force la glibc à réduire son utilisation.

Considérons un processus A et B, où A détient un mutex et B souhaite l'acquérir. 
Observons le comportement de la glibc et du kernel.

Lorsqu'un processus va essayer de prendre un mutex la glibc va faire un test and set atomique sur sa valeur. 
En fonction de cette dernière la gblic déterminera s'il est nécessaire de faire un appel système. 

Dans un premier temps A prend le mutex, sa valeur vaut 0. A peut donc immédiatement l'acquérir et placer
la valeur 1 pour indiquer que le mutex est occupé.

Lorsque B souhaitera prendre le mutex il constatera une valeur à 1, B doit donc s'endormir. Cette opération
nécessite un appel système vers le kernel. Avant l'appel, B prend soin de placer la valeur 2 dans le mutex
afin d'indiquer que des processus attendent sur ce dernier.

Dans l'appel système effectué par B, l'adresse virtuelle de l'espace utilisateur qui contient
la valeur du mutex est passée en paramètre. Cette adresse permet au kernel de récupérer la valeur du mutex, et avec une opération bit à bit, le pid du propriétaire. 
Le kernel utilise l'adresse pour générer et identifier le futex d'une manière unique grâce à une \verb|futex_key|.
Grâce à cela le kernel endort le processus B et le représente par une structure \verb|futex_q| qui attend sur le futex.

Lorsque A relâche le mutex il constatera une valeur à 2, lui indiquant qu'un ou plusieurs processus
attendent sur le mutex. Il est donc nécessaire de faire un appel système pour réveiller ces processus.
Le kernel réveillera les processus associés aux \verb|futex_q|.
Si la valeur du mutex aurait été de 1 alors A aurait placé la valeur 0 sans effectuer d'appel système.

Le processus B est réveillé et deviendra le propriétaire du mutex.

Grâce à ce mécanisme la glibc fait des appels système uniquement si d'autres processus
attendent sur le mutex.

