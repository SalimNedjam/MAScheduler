\section*{Problématique}

Le problème se présente quand plusieurs processus souhaitent accéder à une même
ressource critique. 
En effet, si le processus, qui détient le verrou, empêche un ou plusieurs processus 
d'avancer il est important de le faire passer en priorité dans les 
élections. Ainsi, en favorisant le processus propriétaire du verrou, ce dernier pourra arriver au point de libération de la ressource plus rapidement, et donc de débloquer les autres processus tout en respectant la prévalence des autres processus indépendants pour respecter la 
sémantique du CFS (Completly Fair scheduler).
Il faudra aussi prendre en compte les processus qui abusent de ce mécanisme et 
ceux qui ne relâchent pas le verrou dans un temps fini. 
\\

Pour résoudre ce problème il a fallu dans un premier temps permettre au
kernel de connaître l'identité du processus propriétaire du verrou.
Ensuite le scheduler devait utiliser cette information et agir en conséquence.
\\