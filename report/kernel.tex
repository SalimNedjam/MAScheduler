\section*{Kernel et futex}
La Glibc communique avec le kernel à travers des appels systèmes, elle fournit 
en paramètre l'adresse virtuelle de l'espace utilisateur qui contient la 
valeur du lock.
\\

Le futex associée est identifié avec un futex\_key qui permet d'identifié un futex
d'une maniere unique que ça soit dans un meme espace d'adressage ou bien d'un
segement de mémoire partagé, elle est ensuite utilisée dans un bucket pour lister
tous les processus en attente d'un futex donné.
\\

Expliquer le futex.c et son fonctionnement (du moins ce qui nous intéresse).