\section*{Kernel et futex}
La Glibc communique avec le kernel à travers des appels systèmes. Pour la prise de verrou (futex) elle fournit en paramètre de l'appel système l'adresse virtuelle de l'espace utilisateur qui contient la valeur du lock.

Lorsqu'un processus veut prendre un mutex, ce dernier fait appel à une fonction 
de la Glibc qui essaye, dans un premier temps, de faire un test and set atomique pour essayer d'acquérir le verrou si ce dernier est libre, évitant ainsi de passer inutilement côté kernel. En effet, les appels systèmes sont coûteux en temps, l'objectif pour la Glibc est donc d'en réduire leurs utilisations. Cependant, lorsqu'un thread ou processus souhaite par la suite prendre ce même mutex, préalablement détenu, un passage vers le kernel est nécessaire. L'appel système sera accompagné de l'adresse utilisateur \verb|uaddr| du futex. Le kernel utilise cette adresse pour générer et identifier le futex d'une manière unique grâce à une \verb|futex_key|. Segment de mémoire partagé ou même espace d'adressage, la clé est ensuite utilisée dans un bucket pour avoir accès à une liste de structure \verb|futex_q|. À noter que chaque tâche bloquée en attente d'un futex est représentée par un structure \verb|futex_q|.

Dès lors que le thread détenant le verrou libéra le futex, un nouvel appel système sera nécessaire. Une fois côté kernel, une recherche est faite dans la table de hash, introduite précédemment, afin de trouver les autres tâches en attente du futex. Avant de sortir du kernel le thread réveillera les tâches bloquées. 
\\

\textit{Nous illustrerons la successions des appels de fonction user et kernel mode ici.}
