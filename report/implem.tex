\section{Implémentation}

Notre choix d'implémentation a donc été de ne pas modifier directement le code du scheduler (\verb|<kernel/sched/fair.c>|), mais d'ajouter notre mécanisme lors de la prise de
verrou dans le code
du futex (\verb|<kernel/futex.c>|) et d'agir directement sur les priorités des tâches.

L'ensemble de nos modifications dans le kernel on été préfixés par un commentaire \verb|MAS code|, pour
\verb|Mutex Aware Scheduler|.

\subsection{Structure}

La première étape à donc été de créer un mécanisme permettant au kernel de pouvoir facilement
suivre l'évolution du propriétaire d'un futex.
Nous avons donc ajouté une structure \verb|futex_state| qui représente l'état d'un
futex:

\begin{lstlisting}[tabsize=4]
struct futex_state {
	struct list_head list;
	struct task_struct *owner;
	raw_spinlock_t spin_lock;
	struct kref refcount;
	int load;
	union futex_key *key;
};
\end{lstlisting}

Les champs sont les suivants:
\begin{itemize}
	\item \verb|list| permet de créer un chaînage entre les futex d'un même propriétaire.
	
	\item \verb|owner| est une référence sur la \verb|task_struct| du propriétaire du futex.
	
	\item \verb|spin_lock| permet d'éviter les accès concurrents lors de la manipulation de
	la structure.
	
	\item \verb|kref| est un compteur de référence pour protéger la suppression
	des structures, et ainsi éviter de libérer la structure utilisée ailleurs.
	
	\item \verb|load| est le poids associé au futex.
	
	\item \verb|key| est la clé du futex, permettant d'identifier la structure pour un 
	futex donné.
\end{itemize}
\hspace{1cm}

Une modification sur la structure \verb|task_struct| a été nécessaire:

\begin{lstlisting}[tabsize=4]
struct task_struct {
	...
	struct list_head futex_state_list;
	raw_spinlock_t futex_state_lock;
	struct futex_state *waiting_futex_state;
	int user_nice;
	int futex_state_prio;
	...
}
\end{lstlisting}

Les champs ajoutés sont les suivants:
\begin{itemize}
	\item \verb|futex_state_list| est l'ensemble des
	\verb|futex_state| que la tâche détient et sur lesquels d'autres tâches attendent.
	
	\item \verb|futex_state_lock| est un verrou pour la manipulation de la liste.
	
	\item \verb|waiting_futex_state| est un pointeur vers le \verb|futex_state| sur lequel
	la tâche attend.
	
	\item \verb|user_nice| est le valeur courante du nice utilisateur appliquée à la tâche.
	
	\item \verb|futex_state_prio| est l'augmentation de priorité appliquée à la tâche en fonction
	des autres tâches qu'elles bloquent.
\end{itemize}

\subsection{Fonctionnement}

Lorsqu'une tâche souhaite prendre un verrou déjà détenu elle passe en mode kernel. 
C'est là que notre mécanisme rentre en jeu.

L'appel système place dans \verb|uaddr| l'adresse du verrou côté user. 
La valeur du verrou, contenant le pid du propriétaire, peut être récupérée avec \verb|get_user(uval, uaddr)|.
Ainsi, la tâche qui va se bloquer en attente du verrou connaît le pid du propriétaire avec une simple
opération bit à bit sur \verb|uval| et peut accéder sa structure \verb|task_struct| avec
\verb|find_task_by_vpid(vpid)|.

\subsubsection{Prise de verrou}

Lors de la prise de verrou notre première étape est de récupérer la \verb|futex_key| 
associé au futex sur lequel la tâche va se bloquer:
\begin{lstlisting}[tabsize=4]
	get_futex_key(uaddr, 0, key, VERIFY_READ);
\end{lstlisting}
Ensuite il faut récupérer le \verb|futex_state| associé à cette \verb|futex_key|:
\begin{lstlisting}[tabsize=4]
	get_futex_state(owner, key, &state);
\end{lstlisting}

Cette fonction va regarder si le \verb|futex_state| associé à la clé \verb|key| existe
dans la liste du \verb|owner|,
si la structure n'existe pas alors elle sera allouée, initialisée et ajoutée à la liste du \verb|owner|,
sinon elle sera retournée.
Dans les deux cas le compteur de référence du \verb|state| retourné sera incrémenté.

Pour indiquer que la tâche courante va s'endormir en attendant sur le futex, on passe la structure
du \verb|futex_state| précédemment récupérée à la \verb|task_struct|:
\begin{lstlisting}[tabsize=4]
	current->waiting_futex_state = state;
\end{lstlisting}

\subsubsection{Héritage de charge}

Après l'ajout de la tâche courante comme tâche en attente du futex il faut appliquer le changement de charge
par héritage:
\begin{lstlisting}[tabsize=4]
	futex_state_inherit(current, state, 
						FUTEX_STATE_LOAD);
\end{lstlisting}

La tâche courante \verb|current| peut elle aussi bloquer plusieurs tâches sur plusieurs verrous, ainsi il est important de ne pas
simplement incrémenter de 1 le poids du futex sur lequel on va attendre, mais de la somme des poids
des \verb|futex_state| que la tâche courante détient. Cette somme est calculée avec un simple parcourt sur la
liste des \verb|futex_state| de la tâche:
\begin{lstlisting}[tabsize=4]
	get_futex_state_sumload(task);
\end{lstlisting}

Le propriétaire de \verb|state| peut lui aussi attendre sur un \verb|futex_state|,
détecté grâce au champs \verb|waiting_futex_state|. 
Le principe d'héritage demande donc
de procéder par récurrence sur l'arbre et de mettre à jour le poids de chaque \verb|futex_state|
et ainsi remonter à la tâche qui n'attend pas sur un verrou, le \textit{master futex owner}.
C'est ce dernier qui verra sa priorité changer à partir de son poids.

\begin{lstlisting}[tabsize=4]
int futex_state_inherit(struct task_struct *task, 
						struct futex_state *state,
						int op)
{
	struct futex_state *m_state;
	int sumload;
	
	if (op != FUTEX_STATE_LOAD && 
		op != FUTEX_STATE_UNLOAD)
		return -1;
	
	/* Recupere la somme des poids des eventuelles 
 	   futex_state que l'on detient */
	sumload = get_futex_state_sumload(task);
	
	/* Remonte l'arbre jusqu'au master futex 
 	   owner et applique le poids a chaque futex_state */
	do {
		m_state = state;
		state->load += (sumload + 1) * op;
	} while ((state = state->owner->waiting_futex_state)
		!= NULL);
	
	/* Applique la priorite sur le master futex owner */
	futex_state_prio(m_state->owner);
	
	return 0;
}
\end{lstlisting}

\subsubsection{Application de la priorité}

La fonction \verb|futex_state_prio(task)| permet de définir la priorité d'une tâche en se
basant sur son \verb|sumload|. L'implémentation finale est très simple et peut être la cible
d'amélioration. 
\begin{lstlisting}[tabsize=4]
void futex_state_prio(struct task_struct *task)
{
	int load = get_futex_state_sumload(task);
	
	if (load < 0)
		load = 0;
	
	if (load > FUTEX_STATE_MAX_PRIO)
		load = FUTEX_STATE_MAX_PRIO;
	
	task->futex_state_prio = load;
	set_static_prio(task);
}
\end{lstlisting}

L'augmentation attribuée à la tâche correspond au nombre de tâche qu'elle bloque, borné sur 
\verb|FUTEX_STATE_MAX_PRIO| valant 20.

La fonction \verb|set_static_prio(task)| est ensuite appelé.
Cette fonction se situe dans \verb|<kernel/sched/core.c>| et permet de modifier la priorité
statique d'une tâche en combinant le nice utilisateur (\verb|user_nice|) et l'augmentation
de priorité lié au futex state (\verb|futex_state_prio|).
Voici son code simplifié:

\begin{lstlisting}[tabsize=4]
void set_static_prio(struct task_struct *p)
{
	int prio;
	
	prio = NICE_TO_PRIO(p->user_nice) - p->futex_state_prio;
	
	...
	
	p->static_prio = prio;
	set_load_weight(p, true);
	old_prio = p->prio;
	p->prio = effective_prio(p);
	
	...
}
\end{lstlisting}

La fonction appelée lors de la modification du nice par l'utilisateur a aussi
subi des changements:
\begin{lstlisting}[tabsize=4]
void set_user_nice(struct task_struct *p, long nice)
{
	if (task_nice(p) == nice || nice < MIN_NICE 
		|| nice > MAX_NICE)
	return;
	
	p->user_nice = nice;
	set_static_prio(p);
}
\end{lstlisting} 

Ce mécanisme de changement de priorité permet de faire cohabiter le nice utilisateur
et l'augmentation de priorité des \verb|futex_state|.
Quand l'une des deux valeurs change l'autre est toujours prise en compte.
On peut donc voir la variable \verb|futex_state_prio| comme un offset qui décale la priorité de base
(\verb|DEFAULT_PRIO| = 120) sur lequel le nice base son calcul.



\subsubsection{Changement de propriétaire}

Lorsqu'une tâche détenant un verrou le relâche, celui-ci réveille toutes les autres tâches en attente
du verrou. La tâche qui sera le nouveau propriétaire le déclare avec la fonction:
\begin{lstlisting}[tabsize=4]
	fixup_state_owner_current(state);
\end{lstlisting}
Cette fonction va notamment modifier le \verb|owner| de \verb|state|, décrémenter son poids, ajouter 
le \verb|state| à sa liste, et ainsi pouvoir appliquer sa priorité basé sur ces changements avec
\verb|futex_state_prio(task)|. Le champs \newline \verb|waiting_futex_state| de la tâche courante est marqué à \verb|NULL|
pour indiquer que celle-ci n'attend plus sur un futex. Voici le code de la fonction:

\begin{lstlisting}[tabsize=4]
int fixup_state_owner_current(struct futex_state *state)
{
	int sumload = get_futex_state_sumload(current);
	state->owner = current;
	add_futex_state(state);

	state->load -= (sumload + 1);	

	futex_state_prio(state->owner);
	
	kref_put(&state->refcount, free_futex_state);
	current->waiting_futex_state = NULL;
	
	return 0;
}
\end{lstlisting}

Il faut noter que le poids est décrémenté de 1 plus la somme des poids des \verb|futex_state| 
que le nouveau propriétaire détient avant l'ajout du \verb|state|. En effet, en se bloquant on a hérité notre \verb|sumload| au \verb|state|, en devenant son propriétaire il faut donc déshériter ce même \verb|sumload|.

\subsubsection{Relâchement d'un verrou}

Lors du relâchement d'un verrou la tâche courante, le propriétaire, doit récupérer le \verb|futex_state| associé:
\begin{lstlisting}[tabsize=4]
	fetch_futex_state(current, key, &state);
\end{lstlisting}
Si le \verb|state| existe alors il est retiré de la liste:
\begin{lstlisting}[tabsize=4]
	del_futex_state(state);
\end{lstlisting}
Après ces changements la priorité de la tâche courante doit être mis à jour:
\begin{lstlisting}[tabsize=4]
	futex_state_prio(current);
\end{lstlisting}

Ainsi, si la tâche relâche son dernier verrou alors sa liste devient vide et l'appel
à \verb|futex_state_prio(task)| annulera les modifications de priorité apportées.

\subsubsection{Mort subite}

Il se peut qu'un processus meurt alors qu'il attend sur un verrou. De la même manière que lors de
l'attente sur un verrou, il faut cette fois-ci décrémenter par récurrence sur l'arbre 
le poids des \verb|futex_state|.
Une modification de la fonction \verb|do_exit| dans \verb|<kernel/exit.c>| a donc été nécessaire:
\begin{lstlisting}[tabsize=4]
if (tsk->waiting_futex_state != NULL) {
	futex_state_inherit(tsk, tsk->waiting_futex_state,
						FUTEX_STATE_UNLOAD);
	kref_put(&tsk->waiting_futex_state->refcount, 
			free_futex_state);
}
\end{lstlisting}

Si la tâche courante attend sur un futex on applique donc un deshéritage et on relâche la référence sur le
\verb|futex_state|. 

\subsection{Shrinker}

Les shrinker permettent une optimisation mémoire.
Dans notre cas leur utilisation n'est pas possible. En effet, on ne peut pas se permettre 
de supprimer ou bien de libérer une référence d'une structure \verb|futex_state| en cours 
d'utilisation, cela peut compromettre tout le mécanisme de verrouillage des futex.