\section{Discussion}

Nous concluons ce projet par un bilan sur sa réalisation et ses résultats ainsi qu'une discussion sur 
ses limites et les améliorations envisageables.

\subsection{Conclusion et limite}

La réalisation du projet s'est bien déroulé, toutes les fonctionnalités attendues on été
implémentées, on a pu conclure que dans certains cas de forte contention l'approche du CFS à ces limite car il ne prends pas en compte les thread en attente d'entrée en section critique, Les résultats qu'on a pu tirer de notre mécanisme sont globalement concluant contenu des metrics que l'on s'est fixés au départ.

Cependant le projet a montré des limites, notamment sur la manipulation de la liste \verb|futex_state_list| d'une tâche.
Son utilisation requière la prise du verrou \verb|futex_state_lock|.
Ainsi, lors de la recherche, d'un ajout ou d'une suppression d'un \verb|futex_state|
sur la liste mais aussi lors du calcul de la somme, le verrou est pris. Cela empêche d'autres tâches bloquées
sur la même tâche propriétaire d'y accéder. Créant ainsi des groupes de tâches à cette prise de
verrou commun. Il y a donc un risque de contention entre ces tâches.

Dans les premières implémentations nous avions créé un liste globale, recensant l'ensemble des \verb|futex_state| quelque soit
leur propriétaire. Sa manipulation était donc commune à toutes les tâches, en plus du manque de performance de 
cette méthode, le problème de contention était d'autant plus présent. L'utilisation des listes, dites locales, à chaque
propriétaire permet de réduire ce problème de contention. Cependant le risque persiste et aucune solution 
viable nous semble contourner ce problème pour le moment.

\subsection{Travaux à court terme}
Par manque de temps certaines tâches n'ont pas pu être réalisées dans le temps imparti.
Leur implémentation n'ont pas d'impact majeur et elles ont donc été laissé pour les travaux à court terme.

\subsubsection{Poids}
La clé de voûte de notre implémentation est son système de poids.
Lors de l'héritage nous avons vu que la tâche qui va se bloquer doit calculer la somme
des poids des \verb|futex_state| qu'elle détient, représenté ensuite par une variable \verb|sumload|.
Ce calcul est effectué plusieurs fois, et demande de parcourir la liste des \verb|futex_state| de la tâche.
Or, nous pouvons réduire ces calculs par un champs dans la structure \verb|task_struct| qui représenterai
la valeur \verb|sumload| à tout instant. Cette variable serait mis à jour lors de l'héritage, deshéritage,
ajout et suppression de \verb|futex_state|.

\subsubsection{Généralisation}
Par manque de temps l'implémentation n'a été qu'effectué que pour l'utilisation des mutex de type PI 
(héritage de priorité). Un travail à court terme serait de le rendre
fonctionnel pour l'utilisation des mutex de type ROBUST, soit le protocole par défaut.

Cette décision est une question de débogage, nous permettant 
de tester notre implémentation volontairement avec des programmes tests en utilisant
des mutex PI. En effet, en implémentant le mécanisme, en cours de développement,
 pour le type ROBUST le kernel l'aurait utilisé, ce qui aurait compliqué le débogage lors d'éventuelles erreurs.

\subsection{Travaux à long terme: les priorités}
L'implémentation actuelle du changement de priorité d'une tâche par la fonction \verb|futex_state_prio(task)| 
est très simpliste. Elle réside en une réduction de la \verb|static_prio| par le \verb|sumload| 
borné sur 20.

Pour rendre plus juste la modification de la priorité nous pouvons envisager l'utilisation d'une formule.
Une approche peut consister à attribuer un coefficient à chaque tâche bloquée. 
Ce coefficient est calculé en fonction de la priorité de la tâche. 
Plus la tâche est prioritaire et plus son coefficient est élevé.
Le poids qu'ajoutera la tâche bloquée au \verb|futex_state| du propriétaire dépendra de son coefficient.
Ainsi par exemple, une tâche de priorité 120 ajoutera 1 au poids, tandis qu'une tâche de priorité 100
ajoutera 2. 

Ce système permet d'être encore plus équitable entre les tâches. Deux tâches de même priorité bloquant
le même nombre de tâches auront un \verb|sumload| différent en fonction des priorités des tâches qu'elles bloquent.

Cette nouvelle approche complexifie le mécanisme, notamment si une tâche bloquée voit sa priorité modifiée
lorsqu'elle dort, son coefficient devra être mis à jour en conséquent.

