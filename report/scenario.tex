\section{Scénario}

Dans la suite du rapport nous considérons avoir trois tâches: A, B et C. 
L'objectif présenté par les scénarios est de favoriser
la ou les tâches qui détiennent un verrou pour débloquer rapidement le fil d'exécution
des autres tâches potentiellement en attente, tout en ne dénaturant pas les principes
clés du CFS.

\subsection{Monocœur}

Considérons un monoprocesseur où les tâches A et C ont une priorité de 120 et 
B une priorité de 100. La tâche A détient un verrou.
Le taux de CPU attribué aux tâche A et C sera le même, cependant A doit être
la tâche cible de notre modification, pour que celle-ci soit favorisé lors
de l'élection par rapport à C. Cependant, la modification apportée ne doit pas 
être une gêne pour l'avancement de B, qui est lui de base, est plus prioritaire.

\subsection{Multicœur}

Considérons un processeur avec deux cœurs où les tâches A et B sont des lots
d'exécutions contenant chacun 5 threads, avec la même priorité.
Le taux CPU attribué aux deux tâches est le même. Aucun des threads de A ne prend de verrou.
Au bout d'un certains temps un thread de B prend un verrou, bloquant tous les autres threads de B.
Il faudra 100 \verb|timeslice| pour le thread propriétaire du verrou pour exécuter sa section
critique et ainsi relâcher le verrou, débloquant tous les autres threads de B.
Si les 5 threads de A s'exécutent chacun en 10 \verb|timeslice| alors le thread A se terminera
avant la libération du verrou de B. Ainsi, la tâche A va libérer son cœur et ce dernier n'exécutera
aucune instruction avant que les threads de B ne soit débloqués.
Notre solutions doit donc éviter le gâchis de cœur. En favorisant B ce dernier va exécuter
sa section critique plus rapidement, permettant ainsi de finir avant, ou en même temps, que le thread A.
Lorsque A finira son exécution les autres threads de B, débloqués, pourront utiliser le cœur libéré.

\subsection{Charge}

Notre solution pour influencer la priorité d'une tâche, tout en restant équitable,
est d'introduire un système de charge à chaque verrou.

\subsubsection{Augmentation de la priorité}

Pour reprendre le scénario du monoprocesseur nous pouvons envisager de favoriser la tâche A, détenant un verrou,
en augmentant sa priorité. Cette augmentation ne doit cependant pas être un inconvénient pour B,
plus prioritaire. Une augmentation d'une valeur de 10 est un bon compromis. La tâche A
devient alors plus prioritaire que C sans dépasser B.

Cependant, si C détient lui aussi un verrou, et que A et C bloquent respectivement 
10 et 100 tâches, alors augmenter la priorité de 10 au deux tâches semble non équitable
étant donné que C bloque plus de tâches que A. Ainsi, une augmentation dans un
intervalle de 0 à 10 est plus cohérent. La tâche A aura une priorité augmentée
de 2 contre 10 pour C. Cette augmentation de priorité sera déterminé
avec un système de \textbf{charge}.

Une autre solution aurait pu être de modifier directement la valeur \verb|vruntime|
pour forcer la tâche à être à gauche de l'arbre \verb|rbtree| et ainsi assurer sa
réélection. Cependant, cela implique que la tâche passera devant tous les autres tâches,
y compris celles plus prioritaires. Cette solution écrase toute la logique du système
de priorité mis en place par le scheduler et donc n'a pas été retenue.


\subsubsection{Calcul de charge}

Chaque tâche propriétaire d'un verrou se verra attribuer une charge, celle-ci évoluera en fonction
des tâches bloquées sur le verrou. Plus une tâche a une charge élevée, plus sa priorité augmente.

L'implémentation du calcul de charge est assez simple. 
La charge correspondra au nombre total de tâches couramment bloquées sur le verrou. 
Ainsi, la priorité ajoutée sera dans l'intervalle de 0 à 10: une tâche bloquée augmentera
la priorité de 1, bornant l'augmentation à 10 tâches bloquées.


\subsubsection{Héritage de charge}

Pour rendre le système de charge plus efficace nous introduisons un héritage
de charge entre les tâches. Considérons que B détient un verrou et se bloque
en essayant de prendre un deuxième verrou détenu par A. Alors A verra sa
charge augmenter par la valeur celle de B. Ainsi, la tâche A sera plus favorisée même
si A et B ont la même priorité et la même charge au départ. Si une troisième tâche C,
ne détenant pas de verrou, se bloque sur le verrou détenu par A, alors l'héritage fera incrémenter
de 1 la charge de A et de B.

Dans le même principe, si C meurt avant d'avoir récupéré le verrou, alors un deshéritage est
appliqué: A et B voient leur charge décrémenter de 1.

L'héritage de charge est très utile mais demande aussi d'être très précis.
En effet, si la charge d'une tâche est incorrecte par rapport à la réalité
(i.e. la valeur ne correspond pas au nombre de tâches bloquées)
alors l'héritage ne peut qu'aggraver l'erreur. Il est très important
de s'assurer que la charge est valide en la décrémentant lorsqu'une tâche 
n'attend plus sur le verrou, notamment lorsque la tâche meurt avant de l'avoir pris.
