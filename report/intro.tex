\section*{Introduction}
	L'une des tâches les plus importantes qu'un OS doit faire est de faciliter le
	multi-tasking.
	C'est l'ordonnanceur qui s'occupe de cette tâche critique.
	
	L'ordonnanceur a pour but premier de performer cette fonction le plus
	efficacement possible en utilisant le moins de ressources, et en respectant les
	trois principes qui sont Safety, Liveness et Fairness.
	\\
	
	Les objectifs de notre projet ont donc été les suivants: optimiser les
	décisions de l'ordonnanceur pour apporter une élection plus favorable aux
	processus détenant un verrou.
	\\
	
	Nous verrons tout d'abord dans ce rapport les différents aspects de la Glibc
	qui permet de gérer la synchronisation. 
	
	Ensuite, nous aborderons les différents points qui font le lien entre la Glibc
	et le kernel. 
	
	Nous introduirons le scheduler et son fonctionnement, ainsi que les différents
	scénarios.
	
	Enfin, nous présenterons notre implémentation et les benchmarks.